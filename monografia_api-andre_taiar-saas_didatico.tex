\documentclass{abnt}
\usepackage[brazil]{babel}
\usepackage[utf8]{inputenc}
\usepackage[num]{abntcite}
\usepackage{graphicx}
\usepackage{url}

\graphicspath{{imagens/}}

\begin{document}
\autor{André Taiar Marinho Oliveira}

\titulo{SAAS didático para um sistema de gestão estratégica}

\orientador[Orientador:\\]{Prof. Dr. Clarindo Isaías Pereira da Silva e Pádua}
\coorientador[Orientador:\\]{Prof. Dr. Marcelo Aureliano Monteiro de Andrade}

\comentario{Apresentado como de trabalho na disciplina de Atividades Práticas Integradoras do Curso de Bacharelado em Sistemas de Informação da UFMG}

\instituicao{Universidade Federal de Minas Gerais \par Instituto de Ciências
Exatas \par Departamento de Ciência da Computação}

\local{Belo Horizonte} \data{2015/2}

\capa
\folhaderosto

\begin{resumo}

Este trabalho tem como objetivo fundamentar os alicerces para a construção de um
produto dando embasamento teórico e a iniciação do desenvolvimento de um sistema
de gestão estratégica voltado para pequenas e médias empresas.

Em um sentido mais específico, a fundamentação seria mais precisamente na
integração de algumas ferramentas de gestão estratégica que compartilham entre
si dados comuns no contexto de uma organização. Como parte do projeto de
desenvolvimento do sistema, será preciso também utilizar conceitos de
engenharia de software para especificar e documentar os requisitos e
funcionalidades que serão fundamentais para o desenvolvimento do projeto.

Obteremos no final uma base teórica para guiar o desenvolvimento deste produto,
a especificação inicial do sistema que deseja-se desenvolver e um protótipo
mínimo da solução a ser desenvolvida.

\textbf{Palavras-chaves}: saas, gestão, estratégia, engenharia de software.
\end{resumo}

\sumario %comando que gera o sumário automaticamente
\renewcommand*\listfigurename{LISTA DE FIGURAS}
\listoffigures %comando que gera um sumário para a lista de figuras do texto automaticamente


\chapter{INTRODUÇÃO}

O planejamento estratégico das organizações surgiu em em meados da década de 60
se tratando de uma metodologia que permite estabelecer a direção a ser seguida
pela organização e visa um maior grau de interação com o ambiente aonde ela
atua. É um processo aonde se observa a organização por diversos ângulos,
direcionando os seus rumos e monitorando as suas ações de forma concreta.
Grandes empresas se beneficiam diretamente do planejamento através de gestão
estratégica de suas atividades, seja em um escopo amplo ou em projetos e sub
produtos dentro de sua cadeia de produção.

Micro e pequenas empresas não costumam se beneficiar diretamente do planejamento
estratégico e muitos motivos podem ser os motivos para que isso ocorra. O
assunto de Gestão Estratégica é bastante amplo; muitas técnicas podem ser
utilizadas e diversas análises podem ser feitas em cima de uma mesma organização
e dominar a utilização e entendimento destas ferramentas pode ser bastante
complicado para alguém não especialista na área. Mesmo que um gestor de
organização tenha domínio ou conhecimento específico sobre algumas destas
ferramentas do gerenciamento, utilizá-las parcialmente ou separadamente não é
tão efetivo quanto aplicá-las em uma sequência correta de passos com informações
integradas entre as etapas do processo. Além destes motivos, geralmente o gestor
de pequenas e médias empresas já tem grande parte do seu tempo tomado por
cumprir as funções do próprio metabolismo da organização, faltando tempo a ser
investido em conhecer melhor tais ferramentas e sua correta aplicação dentro do
cenário de sua organização.

A proposta desta monografia é desenvolver o protótipo de um sistema que agrupe
diversas ferramentas de gestão para planejamento estratégico de forma integrada.
A utilização destas ferramentas se dará de tal forma que o usuário do sistema
interaja com a plataforma seguindo um fluxo definido de atividades que se
integram e geram resultados mais relevantes em seu conjunto de especialidades.
Este fluxo de utilização deve ser o mais didático possível, permitindo que um
gestor de organização, independente de sua familiaridade com ferramentas de
gestão estratégica, consiga utilizar corretamente os conceitos e obter
informações para avaliação e direcionamento do negócio.

Espera-se que deste trabalho resulte um protótipo funcional que seja distribuído
como um Software as a Service (Saas) \cite{Turner2003}. A principal
característica em um software como serviço é a não aquisição das licenças mas
sim pagar pelo uso como um serviço. Funcionando em ambiente web, espera-se que
o modelo possibilite a aquisição do produto por um baixo preço, tornando sua
utilização viável para micro e pequenas empresas e atraindo mercado maior devido
a escala que pode alcançar.

\chapter{CONTEXTUALIZAÇÃO E TRABALHOS RELACIONADOS}

A temática deste trabalho é fortemente apoiada sob três pilares principais e
multidisciplinares. Primeiramente, será desenvolvido um sistema e isso será a
parte mais técnica do trabalho e reunirá conceitos de engenharia de software,
cloud computing e gerenciamento de projetos. Em segundo lugar, o software deve
implementar soluções para gerenciamento estratégico, abrangendo ferramentas
baseadas em teorias de gestão estratégica de organizações. Por fim, o sistema e
as informações devem ser disponibilizados por meio de uma interface intuitiva e
com forte apelo didático, apoiando o usuário às informações necessárias para
alimentar o sistema e, em alguns pontos críticos, instruindo o usuário na
operação que deve ser feita no sistema.

\section{Gestão estratégica}

% contextualização da gestão estratégica
O planejamento estratégico é considerado um instrumento administrativo
relacionado à estratégia empresarial, pois é a sustentação do desenvolvimento e
da implementação de estratégias empresariais \cite{OLIVEIRA1991}. 

Por definição, planejamento significa o desenvolvimento de um programa para a
realização de objetivos e metas organizacionais, envolvendo a escolha de um
curso de ação, a decisão antecipada do que deve ser feito, a determinação de
quando e como a ação deve ser realizada \cite{anaTerence}. 

A estratégia diz respeito à utilização dos recursos da organização que estão à
disposição de seu gestor \cite{ansoff1991nova}. Ao adotar uma estratégia, o
gestor analisar sua organização e o ambiente em que esta está inserido, para
decidir quais são os caminhos, os recursos e as ações que devem ser seguidos
para alcançar os objetivos previamente definidos \cite{anaTerence}.

% descrição básica dos módulos selecionados
O que chamamos neste trabalho de \textbf{ferramentas de gestão estratégica} se
refere a métodos, artefatos e regras que compõem um grupos de ferramentas de
gerenciamento de performance estratégica. No escopo do nosso produto, estas
ferramentas podem ser classificadas em seis grupos distintos:

\begin{itemize}
	\item Análise do negócio;
	\item Análise do mercado;
	\item Modelo de negócio;
	\item Planejamento estratégico;
	\item Análise de viabilidade econômica e financeira;
	\item Plano de ação.
\end{itemize}

Em nosso produto, estes grupos formariam a cadeia de análise e planejamento de
estratégia empresarial começando pelo posicionamento do negócio e passando
sequencialmente pelas ferramentas de análise do mercado, modelagem do negócio,
planejamento da estratégia, análise de indicadores e ferramentas de
acompanhamento e plano de ação.

Dentro de cada um destes grupos devem ser disponibilizadas diversas ferramentas.
Enumeraremos abaixo cada grupo e cada ferramenta que inicialmente deveria compor
o grupo.

\subsection{Ferramentas de Análise do negócio} \label{ferramentas}

\begin{itemize}
	\item Diagrama Empresarial \cite{DiagramaEmpresarial}
	\item Curva de Valor \cite{CurvaValor}
	\item Matriz BCG \cite{BCG}
	\item Análise SWOT \cite{SWOT}
	\item Cadeia de Valor \cite{CadeiaDeValor}
\end{itemize}

\subsection{Ferramentas de Análise do mercado}

\begin{itemize}
	\item Forças do Mercado \cite{Porter}
	\item Forças Macroeconômicas \cite{Porter}
	\item Forças da Indústria \cite{Porter}
	\item Tendências principais \cite{Porter}
\end{itemize}

\subsection{Ferramentas de Modelo de negócio}

\begin{itemize}
	\item Canvas Model \cite{Canvas}
	\item Padrões de Modelo de Negócio \cite{ModeloNegocio}
	\item Ambiente de Modelo de Negócio \cite{AmbienteNegocio}
	\item Avaliação de Modelo de Negócio \cite{Canvas}
\end{itemize}


\subsection{Ferramentas de Planejamento estratégico}

\begin{itemize}
	\item Balanced Scorecard \cite{BSC}
	\item Mapa Estratégico \cite{MapaEstrategico}
	\item Gerenciamento por Diretrizes \cite{GerDiret}
\end{itemize}

\subsection{Ferramentas de Análise de viabilidade econômica e financeira}

\begin{itemize}
	\item Projetar retorno sobre investimento \cite{RetInvest}
	\item Indicador de EBITDA \cite{Ebitda}
	\item Indicador de Lucratividade \cite{Lucratividade}
	\item Cálculo de Fluxo de Caixa Projetado \cite{FluxoCaixa}
	\item Cálculo de TIR \cite{TIR}
	\item Cálculo de VPL \cite{VPL}
\end{itemize}

\subsection{Ferramentas de Plano de ação}

\begin{itemize}
	\item 5W2H \cite{5W2H}
	\item Cronograma e Workflow \cite{Crono}\cite{WorkFlow}
	\item OKR \cite{OKR}
	\item Kanban \cite{Kanban}
\end{itemize}

% explicar o gráfico e exemplificar um modelo de fluxo no sistema

\section{Exemplo de integração entre as ferramentas}

O processo de trabalho dentro da plataforma se daria em etapas. Idealmente,
informações geradas em uma ferramenta do estágio anterior podem servir como
entradas para ferramentas de estágios posteriores. Temos como exemplo de um
fluxo possível, o diagrama abaixo:

\begin{figure}[!htb]
	\centering
	\includegraphics[width=\textwidth]{fluxograma_exemplo.pdf}
	\caption{Exemplo de fluxo de trabalho possível no sistema}
	\label{Rotulo}
\end{figure}

À partir de uma ferramenta de análise de negócio como, por exemplo, uma Análise
SWOT, definiríamos algumas estratégias para o nosso negócio. Baseado em tais
estratégias, poderia ser preenchido um \textit{Business Model Canvas} afim de
esboçar ou discutir elementos do negócio e tornar mais claro pontos das
estratégias a serem seguidas. Com estratégias e metas em mãos, poderia ser
montado um \textit{Balanced Scorecard} afim de acompanhar o desenvolvimento das
estratégias baseado em indicadores internos da organização, partindo então para
os planos de ação. A ferramenta \textit{5W2H} serviria para acompanhamento de
diretrizes gerais sobre a forma em que as metas devem ser alcançadas. Tudo
poderia ser metrificado através de ferramentas com Cronogramas e Fluxos de
Trabalho e todo o acompanhamento de tarefas poderia ser verificado através de um
Kanban. Este Kanban, ao final das iterações dos processos, alimentaria novamente
o \textit{Balanced Scorecard} que avaliaria a execução das tarefas. E todo o
clico continua à partir daí até que novas metas entrem no fluxo do sistema.

Um fluxo mais alto-nível é criado em \cite{anaTerence} e é reproduzido na
figura \ref{RoteiroPropostoTerance}.

\begin{figure}[!htb]
	\centering
	\includegraphics[width=\textwidth]{fluxo_empresarial.pdf}
	\caption{Roteiro proposto em \cite{anaTerence}}
	\label{RoteiroPropostoTerance}
\end{figure}

\section{Software (as a Service)}

% contextualizar o software como serviço
O modelo de Software como um Serviço compõe serviços dinamicamente, sob demanda,
integrando diversos outros serviços de mais baixo nível - superando assim muitas
limitações que restringem o uso de software tradicional, sua implantação e
evolução. O foco do Software como um Serviço é separar posso e propriedade de um
software do seu uso. Fornecer funcionalidades de software como um conjunto de
serviços distribuídos que podem ser configurados e utilizados em tempo de
entrega pode superam muitas limitações atuais que restringem o uso do software e
a sua evolução. Esse modelo abre novos mercados, tanto para fornecedores de
serviços específicos de escala relativamente pequena quanto para organizações
maiores que fornecem serviços mais gerais. Ainda existe a possibilidade de
criar e fornecer dinamicamente, serviços inteiramente novos que apenas utilizam
outros serviços já existentes. \cite{dubey2007delivering}

% contextualizar as metodologias que serão utilizadas
Foi vislumbrado que, baseado no conceito de \textit{Software as a Service},
seria desenvolvido um sistema na web que agregaria o uso de todas as ferramentas
cidatas na sessão \ref{ferramentas}. Para apoiar o processo de desenvolvimento
da solução, será utilizado a metodologia Scrum \cite{sutherland2011scrum}.

Em um primeiro momento, foi feito uma sessão de pré-jogo \cite{sutherland2011scrum}
aonde, através de uma conversa entre os \textit{stakeholders} do produto foram
definidas as funcionalidades principais que o sistema precisaria ter. Estes
requisitos foram traduzidos em histórias de usuários e armazenados em um
conjunto de histórias de usuários denominado \textit{backlog} do produto. Nesta
fase do projeto, introduzimos uma figura muito importante que participou
ativamente da fase de especificação e em todas as fases de validação dos
requisitos do projeto, e será referido daqui pra frente como ``dono do
produto''. O dono do produto representa a voz do cliente e é responsável por
garantir que a equipe agregue valor ao negócio. Ele escreve centrado nos itens
do cliente (histórias do usuário), os prioriza e os adiciona para o backlog.

Com o \textit{backlog} do produto priorizado, começou-se a trabalhar em um
conjunto de histórias que dariam o pontapé inicial no projeto, denominado
\textit{Sprint 0} \cite{sutherland2011scrum}. Nesta \textit{Sprint} foram
produzidos alguns artefatos para guiar todo o processo do projeto, tais como
diagramas da arquitetura do sistema e diagramas de casos de uso \cite{sommerville2003engenharia}. 

\section{Didática}

Existem, no Brasil, mais de 3,5 milhões de empresas, das quais 98\% são de
micro e pequeno porte e 80\% dos problemas apresentados por esse tipo de
empresas empresas são de natureza estratégica e apenas 20\% resultam da
insuficiência de recursos.\cite{anaTerence} 

Segundo \cite{almeida1994desenvolvimento}, o processo de planejamento
estratégico nas pequenas empresas deve ser simplificados, pois o pequeno
empresário:

\begin{itemize}
	\item não dispõe de tempo e recursos para realizar um planejamento estratégico
	compelxo;
	\item muitas vezes, não possui adequada formação para realizar as tarefas mais
	complexas do processo;
	\item é imediatista nas suas atividades, exigindo rápido resutlado dos seus
	esforços.
\end{itemize}

Segundo o público alvo que este produto deve atingir, é preciso focar em um
aspecto decisivo que é a facilidade de uso das ferramentas fornecidas. Um fluxo
de utilização, como por exemplo o que foi explicado na figura \ref{Rotulo}, deve
ser algo plausível de ser bem utilizado por um usuário que sequer tenha
conhecimento prévio sobre o uso e interpretação de resultados de qualquer uma
das ferramentas que compõem o compõem.

Não foi feito um estudo rigoroso sobre as melhores maneiras de se abordar este
problema e tudo o que foi pensado durante a elaboração deste trabalho são
impressões individuais provindas de experiência própria como usuário. Além
disso, parece plausível que, preocupar-se com usabilidade em um estágio de
protótipo, aonde sequer um projeto de interface fora definido, soa como uma
otimização prematura a se fazer neste sistema e, por isso, foi decidido que o
mínimo de esforço para viabilizar as soluções seja investido nesta fase do
planejamento.

\chapter{DESENVOLVIMENTO DO TRABALHO}

\section{Proposta e projeto do sistema}

Para o sistema proposto, foi escolhido que se usaria uma arquitetura chamada MVC
(Modelo-visão-controlador) \cite{patternMVC}. Este padrão de arquitetura
organiza o sistema em três componentes:

\begin{itemize}
	\item Modelo: contém as funcionalidades e dados principais;
	\item Visão:  responsável por apresentar os dados ao usuário;
	\item Controlador: trata os eventos de entrada.
\end{itemize}

O MVC separa a apresentação e a interação dos dados do sistema aonde os três
componentes tem responsabilidades distintas mas interagem entre si. Esta
arquitetura é recomendada quando existem várias maneiras de visualizar e
interagir com os dados e são desconhecidos (ou são voláteis) os requisitos de
interação com os dados. Como vantagem, essa arquitetura permite que os dados
sejam alterados de forma independente de sua representação (e vice versa)  
\cite{patternMVC}. Um diagrama de arquitetura referente ao padrão MVC pode ser
visto na figura \ref{MVCC}.

\begin{figure}[!htb]
	\centering
	\includegraphics[width=350px]{mvc.pdf}
	\caption{Padrão de Arquitetura MVC}
	\label{MVCC}
\end{figure}

\section{Desenvolvimento do projeto}

Para desenvolvimento do projeto foi escolhido a linuguagem \textit{Ruby} 
\cite{Ruby} e o framework de desenvolvimento \textit{Ruby on Rails} \cite{Rails}.
O SGDB \cite{SGDB} escolhido foi o PostgreeSQL \cite{Postgree}. Para
prototipação e validação da ferramenta, foi utilizado um fluxo de entrega
contínua. Neste fluxo de desenvolvimento, cada nova funcionalidade agregada à
solução é publicada e implantada num ambiente de homologação e imediatamente
disponibilizada para validação. Para atender a este fluxo foi utilizado o
servidor cloud Heroku \cite{Heroku}. O ambiente de homologação da solução pode
ser acessado em \url{https://apiweb.herokuapp.com/}.

\chapter{RESULTADOS E DISCUSSÃO}

O pilar mais importante na proposta deste trabalho, tanto pelo seu caráter
inovador quanto pela importância crítica que ele tem no produto final, é a
integração das ferramentas com as informações geradas pelos diferentes módulos
do sistema. Esta modelagem se mostrou extremamente complexa. 

Foi altamente complicado especificar módulos desconexos do sistema sem uma
experiência prévia na utilização dos mesmos. Quando partimos para uma abordagem
aonde os módulos se comunicam o problema parece aumentar exponencialmente.
Muitos conceitos tiveram que ser pensados afim de preencher as lacunas entre a
integração dos módulos e a adaptação das informações que eles compartilhariam.

Creio que algum estudo referente à padrões de arquitetura de software
empresarial deva ajudar a vislumbrar uma melhor solução, apoiando-a sobre o
aspecto técnico muito mais do que sobre o aspecto teórico do modelo. Não houve
tempo suficiente para a elaboração dessa pesquisa neste trabalho. 

Desta forma o tempo foi mesmo investido na especificação dos módulos
individualmente, na modelagem alto-nível da integração entre eles e o
desenvolvimento de apenas uma das várias ferramentas mencionadas na sessão 
\ref{ferramentas}, o \textit{Balanced Scorecard}. A implementação desta
ferramenta foi também parcial e não atendeu à toda a especificação definida para
ela. Ainda faltaram algumas histórias de usuários que compunham toda a solução
no backlog e que ainda não conseguiram ser desenvolvidas. As funcionalidades e
todas as regras desta ferramenta se mostraram realmente complexas.

\chapter{CONCLUSÕES E TRABALHO FUTUROS}

O projeto de interação entre os módulos do sistema, de forma que resultados
obtidos do processamento dos dados informados para módulos de etapas iniciais do
projeto fossem utilizados como dados de entrada para módulos mais
posteriores na utilização da plataforma, demonstrou-se com o tempo ser
extremamente mais oneroso do que o que foi estimado no início do projeto. Também
é visível que o plano inicial de desenvolver muitos módulos e ferramentas de
gestão seria inviável ao longo de um semestre letivo. Neste ponto, a metodologia
ágil que foi escolhida para gerenciamento do projeto ajudou a equipe a refinar
as os requisitos que existiam no \textit{backlog} do produto, priorizando o que
provavelmente agregaria mais valor ao projeto em um curto prazo e que poderia
ser relevante no contexto de um sistema de gestão estratégica: o módulo de
\textit{Balanced Scorecard}.

A ferramenta de \textit{Balanced Scorecard} tem um papel bastante importante no
desenvolvimento desta solução pois sempre dá a visão geral das metas da
organização, de como elas estão se desenvolvendo e se o cronograma definido está
sendo cumprido. Por isso, foi eleita a ferramenta que teria a maior atenção,
maior detalhamento e maior ênfase no início do desenvolvimento do sistema.

Um produto mínimo viável como o que foi idealizado nesse trabalho parece ter um
potencial de crescimento muito grande. Por se tratar de um projeto com um certo
apelo comercial, pretendo continuar desenvolvendo o produto e investir mais
esforço na produção de um sistema completo. O objetivo contínuo sempre será
combinar requisitos neste produto e chegar o mais perto possível do Produto
Mínimo Viável, aonde partiremos para a monetização e comercialização dessa
solução.

\bibliography{bibliog}

\end{document}
