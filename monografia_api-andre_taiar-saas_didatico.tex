%Observação importante: para usar os recursos da classe ABNTEX, é necessário fazer o download de sua biblioteca,
%visto que a mesma não é nativa do ambiente \LaTeX.
%Em um distribuição linux baseada em debian o pacote pode ser instalado pelo comando: sudo apt-get install abntex.
%Autor:Jean Henrique Ferreira Freire

\documentclass{abnt}
\usepackage[brazil]{babel}
\usepackage[utf8]{inputenc}
\usepackage[num]{abntcite}
\usepackage{graphicx}
\usepackage{url}

\begin{document}
\autor{André Taiar Marinho Olveira}

\titulo{SAAS didático para um sistema de gestão estratégica}

\orientador[Orientador:\\]{Prof. Dr. Clarindo Isaías Pereira da Silva e Pádua}
\coorientador[Orientador:\\]{Prof. Dr. Marcelo Aureliano Monteiro de Andrade}

\comentario{Apresentado como de trabalho na disciplina de Atividades Práticas Integradoras do Curso de Bacharelado em Sistemas de Informação da UFMG}

\instituicao{Universidade Federal de Minas Gerais \par Instituto de Ciências
Exatas \par Departamento de Ciência da Computação}

\local{Belo Horizonte} \data{2015/2}

\capa 
\folhaderosto 


\begin{resumo}

Lorem ipsum dolor sit amet, consectetur adipisicing elit, sed do eiusmod
tempor incididunt ut labore et dolore magna aliqua. Ut enim ad minim veniam,
quis nostrud exercitation ullamco laboris nisi ut aliquip ex ea commodo
consequat. Duis aute irure dolor in reprehenderit in voluptate velit esse
cillum dolore eu fugiat nulla pariatur. Excepteur sint occaecat cupidatat non
proident, sunt in culpa qui officia deserunt mollit anim id est laborum.\\


\textbf{Palavras-chaves}: saas, gestão, estratégia. 
\end{resumo}

\sumario %comando que gera o sumário automaticamente
\renewcommand*\listfigurename{LISTA DE FIGURAS}
\listoffigures %comando que gera um sumário para a lista de figuras do texto automaticamente



\chapter{INTRODUÇÃO}

O planejamento estratégico das organizações surgiu em em meados da década de 60,se
tratando de uma metodologia que permite estabelecer a direção a ser seguida pela
organização, visando um maior grau de interação com o ambiente aonde ela atua. É um
processo aonde se observa a organização por diversos ângulos, direcionando os seus rumos e
monitorando as suas ações de forma concreta. Grandes empresas se beneficiam diretamente
do planejamento através de gestão estratégica de suas atividades, seja em um escopo amplo ou
em projetos e sub produtos dentro de sua cadeia de produção.

Micro e pequenas empresas não costumam se beneficiar diretamente do planejamento
estratégico e os motivos são muitos: desconhecimento das ferramentas de gestão, falta de
conhecimento para utilização das ferramentas de gestão, aplicação errada das etapas do
planejamento estratégico, falta de dinheiro para investimento em consultoria, falta de dinheiro
para investimento em ferramentas de gerenciamento etc.

A proposta desta monografia é desenvolver um protótipo de um sistema que agrupe as
ferramentas de gestão para planejamento estratégico e planos de ação em um fluxo lógico
com orientações didáticas e instruções para que um gestor, independente de sua familiaridade
com ferramentas de gestão estratégica, consiga utilizar corretamente os conceitos e obter
informações para avaliação e direcionamento do negócio.

Espera-se que deste trabalho resulte um protótipo funcional que seja distribuído como um
Software as a service (Saas). A principal característica em um software como serviço é a não
aquisição das licenças mas sim pagar pelo uso como um serviço. Funcionando em ambiente
web, espera-se que o modelo possibilite a aquisição do produto por um baixo preço, tornando
sua utilização viável para micro e pequenas empresas.\\

\chapter{CONTEXTUALIZAÇÃO E TRABALHOS RELACIONADOS}
\chapter{DESENVOLVIMENTO DO TRABALHO}
\chapter{RESULTADOS E DISCUSSÃO}
\chapter{CONCLUSÕES E TRABALHO FUTUROS}

\bibliography{teste}

\end{document}
