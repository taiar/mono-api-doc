%Observação importante: para usar os recursos da classe ABNTEX, é necessário fazer o download de sua biblioteca,
%visto que a mesma não é nativa do ambiente \LaTeX.
%Em um distribuição linux baseada em debian o pacote pode ser instalado pelo comando: sudo apt-get install abntex.
%Autor:Jean Henrique Ferreira Freire

\documentclass{abnt}
\usepackage[brazil]{babel}
\usepackage[utf8]{inputenc}
\usepackage[num]{abntcite}
\usepackage{graphicx}
\usepackage{url}

\begin{document}
\autor{Aluno mais aplicado do mundo}

\titulo{Um sistema muito bonito}

\orientador[Orientadores:\\]{YY - Departamento de Engenharia Elétrica}

\coorientador{XX - Departamento de Ciência da Computação}


\comentario{Apresentado como requisito da disciplina de Monografia em Sistemas de Informação do DCC/UFMG}

\instituicao{Universidade Federal de Minas Gerais \par Instituto de Ciências
Exatas \par Departamento de Ciência da Computação}

\local{Belo Horizonte} \data{2013/1}

\capa 
\folhaderosto 


\begin{resumo}
Coloque aqui seu resumo.\\


\textbf{Palavras-chaves}: palavras, chave. 
\end{resumo}

\sumario %comando que gera o sumário automaticamente
\renewcommand*\listfigurename{LISTA DE FIGURAS}
\listoffigures %comando que gera um sumário para a lista de figuras do texto automaticamente



\chapter{INTRODUÇÃO}

ja dizia \cite{Taiar2015}

\chapter{CONTEXTUALIZAÇÃO E TRABALHOS RELACIONADOS}
\chapter{DESENVOLVIMENTO DO TRABALHO}
\chapter{RESULTADOS E DISCUSSÃO}
\chapter{CONCLUSÕES E TRABALHO FUTUROS}

\bibliography{teste}

\end{document}
